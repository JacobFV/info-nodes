\We should be clear: \our objective is not ``can \we make artificial general intelligence?'' but ``how general \textit{can} \we make artificial intelligence?''. The strictest definition of ``general'' intelligence is intractable; for every pattern recognizer, there exists a pattern it cannot recognize. (CITE AGI paper on my iPad) Nonetheless, natural and artificial approaches to the challange informally provide us an intuition for the illusion and, more importantly, direction to follow.

Leaving the theoretical world however, the reality is that the universe is \textit{not} a white noise genorator. At every level of organization, elementary and emergent isomorpisms, symmetries, congruencies, and motifs are perceived interacting in an orderly manner, and the intrinsic motivation to science is that these patterns \textit{can} be understood. The information system required to understand these patterns may not require so much data but rather, a few well chosen priors that reasonably align with the general distribution of data. If a grand unified theory of physics proves tractable, maybe artificial general intelligence is not so impossible after all.
    
This work examines some of the principles underlying intelligence and consolidates them into a working implementation of open-ended artificial `general' intelligence together with experiments and ablation studies. \Our paper is organized as follows: section 2 reviews key principles of intelligence; section 3 describes their novel composition: \pgi (\PGI); section 4 presents one continuous experiment over a diverse set of open-ended learning environments, numerous close-ended tasks and benchmarks, and ablation studies; and finally, section 5 gives a general discussion of this work along with its broader impact, future work, and a conclusion.   
